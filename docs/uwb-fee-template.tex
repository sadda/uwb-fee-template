% !TeX program = pdflatex
% !TEX encoding = UTF-8

\documentclass[11pt, aspectratio=169]{beamer}

\usepackage[sfdefault]{roboto}
\usepackage[T1]{fontenc}
\usepackage[czech, english]{babel}
\usepackage[utf8]{inputenc}
\usepackage{graphicx}
\usepackage{tikz}
\usepackage{hyperref}
\hypersetup{colorlinks=true, linkcolor=fel_blue}

% Uncomment to use Czech/English in slide backgrounds
\def\cslang{}

% Uncomment to use FEL/RICE background
%\def\ricemode{}

% Uncomment to remove the bottom-left FEL text
%\def\nologo{}

\mode<presentation>{\usepackage{./style/beamerthemeRice}}
\setbeamertemplate{navigation symbols}{}

% Setup metadata, all but title and author may be commented out
\title[]{Západočeská Univerzita v Plzni \\ Fakulta Elektrotechnická}
\author{Lukáš Adam}
\date{14. 5. 2025}
\subtitle{Návod pro šablonu prezentace v Latexu}
\mail{adamluk3@fel.zcu.cz}
%\phone{+420 000 000 000}
%\web{www.myweb.com}

%%%%%%%%%%%%%%%%%%%%%%%%%%%%%%%%%%%%%%%%%%%%%%%%%%%%%%%%%%%%%%%%%%%%%%%%%%%%%%%
%
% Here you can start write your presentation
%
%%%%%%%%%%%%%%%%%%%%%%%%%%%%%%%%%%%%%%%%%%%%%%%%%%%%%%%%%%%%%%%%%%%%%%%%%%%%%%%

\begin{document}



\begin{frame}
	\maketitle
\end{frame}



\begin{frame}{Instalace + Návrhy vylepšení}
Existují dvě možnosti instalace šablony:
\begin{itemize}
\item Stáhnout template na \href{https://github.com/sadda/uwb-fee-template}{Github} stránkách.
\item Otevřít si projekt v \href{https://www.overleaf.com/read/mqszxcxvgjcj\#620afc}{Overleafu} a projekt zkopírovat (Menu $\rightarrow$ Create a copy).
\end{itemize}
V obou případech se edituje soubor \texttt{main.tex}.

\vspace{5mm}

Pro dotazy použijte diskuzní fórum na výše zmíněném Github úložišti. Pro reportování chyb nebo zlepšení funkčnosti šablony založte issue nebo pull request tamtéž.
\end{frame}



\begin{frame}{Použití šablony}
Šablona je dokumentována přímo v \texttt{main.tex} souboru. Šablona obsahuje tři parametry, která mohou být od- nebo zakomentovány.
\begin{itemize}
\item cslang (defaultně vypnuto): Ukazuje titulní a závěrečnou stranu v angličtině. Při zapnutí je ukazuje v češtině.
\item ricemode (defaultně vypnuto): Ukazuje titulní a závěrečnou stranu s obrázkem FEL. Při zapnutí ukazuje obrázek RICE.
\item nologo (defaultně vypnuto): Ukazuje logo na levé dolní části každého slidu. Při zapnutí je logo vypnuto.
\end{itemize}
\end{frame}



\begin{frame}
   \usebeamertemplate{last page}
\end{frame}



\end{document}


