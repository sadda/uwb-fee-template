% !TeX program = xelatex
% !TEX encoding = UTF-8

% MODIFIED by Jonathan Kew, 2008-07-06
% MODIFIED by Tomas Kosan, Vaclav Smidl, 2023-04-29

\documentclass[11pt, aspectratio=169]{beamer}
%\documentclass[handout,11pt,aspectratio=169]{beamer}

%\usepackage{fontspec}
%\setmainfont[
% BoldFont={style/fonts/LiberationSansNarrow-Bold.ttf}, 
 %I%talicFont={style/fonts/LiberationSansNarrow-Italic.ttf},
 %BoldItalicFont={style/fonts/LiberationSansNarrow-BoldItalic.ttf}
%]{style/fonts/LiberationSansNarrow-Regular.ttf}
\usepackage[sfdefault]{roboto}  %% Option 'sfdefault' only if the base font of the document is to be sans serif
\usepackage[T1]{fontenc}

% Copyright 2004 by Till Tantau <tantau@users.sourceforge.net>.
%
% In principle, this file can be redistributed and/or modified under the terms of the GNU Public License, version 2.
%
% However, this file is supposed to be a template to be modified for your own needs. For this reason, if you use this file as a template and not specifically distribute it as part of a another package/program, I grant the extra permission to freely copy and modify this file as you see fit and even to delete this copyright notice.

\geometry{papersize={169.5mm,95.5mm}}
\usepackage[czech, english]{babel}
\usepackage[utf8]{inputenc}
\usepackage{graphicx}
% import of eps, pdflatex must be called as: pdflatex -shell-escape to allow eps to pdf autoconvert
\usepackage{epstopdf}
% setup path where graphic is
\graphicspath{{images/}}
\usepackage{xcolor}
\usepackage{hyperref}
\usepackage{pgfpages}

% TIKZ/PGF setup
\usepackage{tikz}
\usepackage{circuitikz}
\usepackage{textcomp}
\usepackage{verbatim}
\usepackage{pgfplots}
\usetikzlibrary{positioning}
\usetikzlibrary{automata}
\usetikzlibrary{arrows, shapes}
\usetikzlibrary{decorations.markings}
\usetikzlibrary{calc}
\usetikzlibrary{shapes.arrows,decorations,scopes}

\usetikzlibrary{shapes.geometric}
\usetikzlibrary{shapes.arrows}
\usepackage{array}

\hypersetup{pdfstartview={Fit}} % fits the presentation to the window when first displayed

%%%%%%%%%%%%%%%%%%%%%%%%%%%%%%%%%%%%%%%%%%%%%%%%%%%%%%%%%%%%%%%%%%%%%%%%%%%%%%%%%%%%%%%%%%%%%%%%
%
% Start with user settings
%
%%%%%%%%%%%%%%%%%%%%%%%%%%%%%%%%%%%%%%%%%%%%%%%%%%%%%%%%%%%%%%%%%%%%%%%%%%%%%%%%%%%%%%%%%%%%%%%%

% setup language, uncomment to use czech language in slide backgrounds also changes Thanks to string to czech
%\def\cslang{}

% setup department, uncomment when RICE instead of FEL has to be used
\def\ricemode{}

% setup clear right-bottom side. uncomment to remove fel logo
\def\nologo{}

% load beamer template
\mode<presentation>
{
    \usepackage{./style/beamerthemeRice}
    %\setbeamercovered{transparent}
}

\title{HABILITATION LECTURE}
\subtitle{Introduction of the applicant}
\author{Lukáš Adam}
\date{April 23, 2025}
\mail{adamluk3@fel.zcu.cz}
\web{https://sites.google.com/view/lukasadam}

% If you wish to uncover everything in a step-wise fashion, uncomment the following command:

%\beamerdefaultoverlayspecification{<+->}

% uncomment if you want to add some notes to your presentation
% they will be displayed on second screen
% then use pympress to show presentation
%\setbeameroption{show notes on second screen=right}



\usepackage{amsmath,amssymb}
\usepackage{mathpazo}

\newenvironment{wideitemize}{\itemize\addtolength{\itemsep}{10pt}}{\enditemize}



\begin{document}

\setbeamertemplate{navigation symbols}{}

\begin{frame}
	\maketitle
\end{frame}


\begin{frame}{EDUCATION AND EMPLOYMENT}
\begin{wideitemize}
\item Education
\begin{itemize}
\item 2011-2015: PhD in \textbf{applied mathematics}, Faculty of Mathematics and Physics, Charles University, Czech Republic.
\end{itemize}
\pause \item Employment
\begin{itemize}
\item 2012-2013: \textbf{University of West Bohemia}, Pilsen, Czech Republic.
\item 2013-2020: Czech Academy of Sciences, Prague, Czech Republic.
\pause \item 2015-2017: Humboldt University, \textbf{Berlin, Germany}.
\item 2017-2020: Southern University of Science and Technology, \textbf{Shenzhen, China}.
\pause \item 2020-2023: Czech Technical University, Prague, Czech Republic.
\item 2024-now: \textbf{University of West Bohemia}, Pilsen, Czech Republic.
\begin{itemize}
\item RICE, Industrial Informatics.
\end{itemize}
\end{itemize}
\end{wideitemize}
\end{frame}


\begin{frame}{WHY RICE, ZČU?}
\begin{wideitemize}
\item Reasons for me
\begin{itemize}
\item High opinion of RICE and Václav Šmídl based on history from 2012.
\pause \item Possibility to get results into real-world production. Cooperation with companies (since 2024 meetings with IKEM, Škoda Digital, Sigma, ČEPS, Ansys).
\pause \item Multi-disciplinary research. Papers with mathematicians, engineers, computer scientists or biologists.
\end{itemize}
\pause \item Reasons for RICE, ZČU
\begin{itemize}
\item Good research record with 5 years of employment abroad.
\item Experience with teaching and student supervision (including PhD students).
\pause \item \textbf{Different background}. Knowledge of theoretical mathematical optimization (ČEPS), computer vision (IKEM, Chráníme mořské želvy), or machine learning (Sigma). My task is to design a numerical optimization method tailored to a specific engineering problem.
\end{itemize}
\end{wideitemize}
\end{frame}


\transition{Research achievements}


\begin{frame}{PUBLICATIONS}
\begin{minipage}[t]{0.53\textwidth}
\begin{wideitemize}
\uncover<1->{\item 7 D1 publications
\begin{itemize}
\item 4$\times$ IEEE Transactions on Industrial Electronics
\item 1$\times$ Omega - International Journal of Management Science
\item 1$\times$ Mathematical Programming
\item 1$\times$ Environmental Modelling \& Software
\end{itemize}}
\uncover<2->{\item 6 Q1 publications, 1 A* conference, 1~A*~conference workshop.}
\uncover<3->{\item International collaboration: MIT, Queen Mary University of London, Inria, Universidade de Vigo, $\dots$}
\uncover<4->{\item H-index 10/9, 319/253 citations (Scopus/WoS)}
\end{wideitemize}
\end{minipage}
\hfill
\begin{minipage}[t]{0.45\textwidth}
\uncover<1->{\includegraphics[width=\textwidth]{images/Paper3.png}
\includegraphics[width=\textwidth]{images/Paper1.png}
\includegraphics[width=\textwidth]{images/Paper4.png}}
\end{minipage}
\end{frame}


\begin{frame}{AWARDS}
\begin{wideitemize}
\item 2024: \textbf{Best paper - Applications} at the WACV 2024 conference (rank A). Our paper on WildlifeDatasets was selected as the best of the 2000 submitted papers.
\pause \item 2018: Presidential excellence postdoctoral fund. Southern University of Science and Technology.
\item 2018: Second place. Best publication at \'UTIA, Czech Academy of Sciences.
\item 2015: Winner. Best publication at \'UTIA, Czech Academy of Sciences, young scientists.
\item 2013: Second place. Best student scientific research in Theoretical Economics.
\item 2011: Second place. Best student thesis at the Department of Probability and Mathematical Statistics.
\end{wideitemize}
\end{frame}


\begin{frame}{PROJECTS}
\begin{wideitemize}
\item Principal investigator (National Natural Science Foundation of China)
\begin{itemize}
\item Research Fund for International Young Students (660k CZK).
\end{itemize}
\pause \item Team member (international projects, National Centres of Competence and OP) % Operational Programmes
\begin{itemize}
\item Norwegian Financial Mechanism: Source-Term Determination of Radionuclide Releases by Inverse Atmospheric Dispersion Modelling, \'UTIA, Czech Academy of Sciences.
\item NCK: Národní centrum pro energetiku II, ZČU. % https://starfos.tacr.cz/projekty/TN02000025
\item NCK: Centrum pokročilých jaderných technologií II, ZČU. % https://starfos.tacr.cz/en/projekty/TN02000012
\item OP VVV: Výzkumné centrum informatiky, ČVUT. % VVV = Výzkum, vývoj, vzdělávání
%\item OP JAK: Český inkubátor technologií pro energetické sítě, ZČU.
\item OP JAK: Nové technologie pro čistou mobilitu, ZČU.
\end{itemize}
\pause \item Team member (national projects)
\begin{itemize}
\item GAČR: Analýza stability optim a ekvilibrií v ekonomii, ÚTIA.
\item GAČR: Variační a numerická analýza v nehladké mechanice kontinua, ÚTIA.
\end{itemize}
\end{wideitemize}
\end{frame}


\transition{Teaching achievements}


\begin{frame}{TEACHING}
\begin{minipage}[t]{0.4\textwidth}
\begin{wideitemize}
\item I started the seminar ``Mathematical Problems of Non-Mathematicians'' at MFF.
\item Currently, accredited subject at ČVUT, UK, TUL and JČU.
\item It was also streamed to ZČU in the past.
\end{wideitemize}
\end{minipage}
\hfill
\begin{minipage}[t]{0.55\textwidth}
\begin{figure}
\includegraphics[width=\textwidth]{images/MPN.jpg}
\end{figure}
\end{minipage}
\end{frame}


\begin{frame}{TEACHING}
\begin{wideitemize}
\item ZČU (FEL):
\begin{itemize}
\item Aplikace výpočetních metod, lecture+seminar.
\end{itemize}
\pause \item ČVUT (FEL+FJFI):
\begin{itemize}
\item Julia for optimization and learning, guarantor, lecture+seminar. Lecture notes (84 stars on Github)
\begin{figure}
\includegraphics[width=0.8\textwidth]{images/Julia.png}
\end{figure}
\item Optimization, lecture.
\end{itemize}
\pause \item UK (MFF):
\begin{itemize}
\item Mathematical problems of non-mathematicians, guarantor, seminar.
\item Introduction to optimization, seminar.
\end{itemize}
\end{wideitemize}
\end{frame}


\begin{frame}{STUDENT SUPERVISION}
\begin{wideitemize}
\item PhD students:
\begin{itemize}
\item Vojtěch Čermák: Fine-grained recognition of animals in the wild. 2025 (submitted), ČVUT. 
\item Václav Mácha: General framework for classification at the top. 2023, ČVUT.
\end{itemize}
\pause \item Master students:
\begin{itemize}
\item Sarah Maya Žídek: Rozpoznávání mořských želv na Srí Lance. In progress, MU. 
\item Pavel Jakš: Moderní metody robustního strojového učení. 2025, ČVUT.
\item Miloš Drobný: Optimalizační úlohy s pravděpodobnostními omezeními. 2018, MFF.
\end{itemize}
\pause \item Bachelor students:
\begin{itemize}
\item Václav Hudeček: Detekce klíčových bodů pomocí metody SIFT. 2023, ČVUT.
\item Pavel Jakš: Robustní strojové učení a adversariální vzorky. 2022, ČVUT.
\end{itemize}
\pause \item Member of multiple Final state exams committee at FEL, ČVUT.
\end{wideitemize}
\end{frame}


\transition{Service to the Community}


\begin{frame}{CONTRIBUTION TO AI ECOSYSTEM}
\begin{minipage}[t]{0.63\textwidth}
\begin{wideitemize}
\uncover<1->{\item Developed multiple tools for animal re-identification. All results are open-source (datasets, machine learning models, competitions and Python packages).}
\uncover<2->{\item Software is extremely popular.

\includegraphics[height=2cm]{images/WD4.png}
\includegraphics[height=2cm]{images/WD1.png}
\includegraphics[height=1cm]{images/WD2.png}}
\uncover<3->{\item Application of computer vision and programming skills to the biological domain.}
\end{wideitemize}
\end{minipage}
\hfill
\begin{minipage}[t]{0.34\textwidth}
\uncover<1->{
\includegraphics[width=\textwidth]{images/datasets-logo.png}
\includegraphics[width=\textwidth]{images/tools-logo.png}
\includegraphics[width=\textwidth]{images/wildlifereID10k-logo.png}
\includegraphics[width=\textwidth]{images/megadescriptor-logo.png}
}
\end{minipage}
\end{frame}

\begin{frame}{OTHERS}
\begin{wideitemize}
\item Workshop and competition organization:
\begin{itemize}
\item Co-organiser of a tutorial at CVPR (rank A*).
\item Organiser of five small international scientific workshops.
\item Competition organization: 142 participants worldwide.
\end{itemize}
\pause \item Reviews: multiple reviews per year (D1 journals, A* conferences, $\dots$).
\pause \item Several talks at Seminář současné matematiky, FJFI, ČVUT.
\item Field operations for monitoring sea turtles (Greece) and lynxes (Czech Republic).
\item Member of the Student Chamber of the Academic Senate, MFF, UK.
\item Commitee head for Študentská vedecká a odborná činnosť (ŠVOČ) in mathematics and informatics.
\end{wideitemize}
\end{frame}


\transition{Selected research Topics}

\begin{comment}
\begin{frame}{Past topics}
\begin{wideitemize}
\item Cooperation with ZČU: optimal control of drives (papers 2017, 2017, 2023), electrostatic separator (2014), DC overhead wire circuit modelling (2021), electromagnetic valves.
\pause \item Mathematical optimization: nonsmooth optimization, stochastic optimization, optimization with PDEs, evolutionary optimization, classification at the top.
\item Machine learning: accuracy at the top.
\item Game theory: coalitional games.
\end{wideitemize}
\end{frame}
\end{comment}


\begin{frame}{OPTIMAL CONTROL OF CONVERTERS AND DRIVES IN REAL TIME}
%\vspace{-10mm}
\begin{figure}
\centering
\includegraphics[height=0.63\textheight]{images/Picture3.jpg}
\hspace{1mm}
\includegraphics[height=0.63\textheight]{images/obr_prototyp.jpg}
\end{figure}
\end{frame}


\begin{frame}{ADVANCED DATA ANALYSIS FOR ELECTRICAL GRIDS - ČEPS}
\begin{figure}
\centering
\includegraphics[height=0.8\textheight]{images/sit.png}
\end{figure}
\end{frame}


\begin{frame}{MACHINE LEARNING FOR PARAMETER DESIGN OPTIMIZATION - SIGMA}
\hspace{-10mm}
\begin{minipage}[t]{0.77\textwidth}
\begin{figure}
\centering
\includegraphics[width=1\textwidth]{images/Picture1.png}
\end{figure}
\end{minipage}
\hfill
\begin{minipage}[t]{0.28\textwidth}
\vspace{15mm}
\begin{figure}
\centering
\includegraphics[height=0.45\textheight]{images/Motor.png}
\phantom{.}\vspace{-73mm}
\includegraphics[height=0.45\textheight]{images/Picture2.png}
\end{figure}
\end{minipage}
\end{frame}


\transition{Future directions}


\begin{frame}{FUTURE DIRECTIONS}
\begin{wideitemize}
\item Continue using tools of mathematical optimization to solve engineering problems.
\item Research: Optimization techniques + machine learning + data analysis
\begin{itemize}
\item Optimal control of converters and drives in real time.
\item Machine learning for parameter design optimization - \textbf{Sigma}.
\item Advanced data analysis for electrical grids - \textbf{ČEPS}.
\pause \item Analyzing feasibility of automatic train operation systems - \textbf{Škoda digital}.
\end{itemize}
\pause \item Research: Computer vision
\begin{itemize}
\item Wildlife re-identification including dataset zoo, training tools and trained models.
\pause \item Islet cell separator for pancreas transplantation - \textbf{IKEM}.
\end{itemize}
\pause \item Teaching
\begin{itemize}
\item Start a (PhD) course on optimization and machine learning.
\pause \item Participate in creating the discussed study programme Industrial Informatics.
\end{itemize}
\end{wideitemize}
\end{frame}


\begin{frame}
   \usebeamertemplate{last page}
\end{frame}

\end{document}


